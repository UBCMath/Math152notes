%%%%%%%%%%%%%%%%%%%%%%%%%%%%%%%%%%%%%%%%%%%%%%%%%%%%%%%%%%%%%%%%%%%%%%%%%%%%%%%
%%%%%%%%%%%%%   Macros for figure insertion
%%%%%%%%%%%%%%%%%%%%%%%%%%%%%%%%%%%%%%%%%%%%%%%%%%%%%%%%%%%%%%%%%%%%%%%%%%%%%%%
%%%%%%%
%%%%%%%  The two main figure insertion macros are
%%%%%%%
%                \figput{<filename w/o extension>}
%                \figplace{<filename w/o extension>}{<hor shift>}{<vert shift>}
%%%%%%%
%%%%%%%  The first just inserts the figure at the current location. The
%%%%%%%  second inserts the figure at the current location but then shifts 
%%%%%%%  horizontally by the second argument and vertically by the third.
%%%%%%%
%%%%%%%  Some typical TeX commands for inserting figures are
%%%%%%%      \centerline{\figput{<filename w/o extension>}}
%%%%%%%      \vadjust{\centerline{\figput{<filename w/o extension>}}}
%%%%%%%      \midinsert\centerline{\figput{<filename w/o extension>}}\endinsert
%%%%%%%      \topinsert\centerline{\figput{<filename w/o extension>}}\endinsert




%%%%%%%
%%%%%%%   TO SET A FIGURE DIRECTORY INSERT, FOR EXAMPLE,
%%%%%%%                 \def\figdir{figures/}
%%%%%%%   IN YOUR SOURCE FILE. REMEMBER THE TAILING /
%%%%%%%



%%%%%%%
%%%%%%%     SELECT (a) YOUR POSTSCRIPT FILE SUFFIX AND (b) YOUR SYSTEM  NOW!
%%%%%%%
\def\suffix{ps}
\newcount\system
%\global\system=1   % for textures 
%\global\system=2   % for msdos
\global\system=3   % for unix(dvips)
%\global\system=4   % for unix(dvips) scaled by a factor of 1.2
%\global\system=6   % for xdvik



\def\ifundefined#1{\expandafter\ifx\csname#1\endcsname\relax}
\ifundefined{figdir}\def\figdir{}\fi
%
% Now for the definitions and main macro for figure inclusion.
%
\newcount\firstline
\newdimen\pswidth  \newdimen\xleft
\newdimen\psheight \newdimen\ytop \newdimen\ybot
\newcount\justx \newcount\justy
\global\justx=0 \global\justy=0
\newdimen\vpos \newtoks\labeL 
\newread\labeLfile \newdimen\xcoord \newdimen\ycoord
\newif\ifdoit 
\newbox\labox
%  variables for use with xdvik
\newdimen\xdvikwid 
\newdimen\xdvikht
\newdimen\pspoints
\newdimen\rwi
\pspoints=1bp
%
\newcount\temp
\def\readdim#1{\global\read\labeLfile to \temp
\global #1=\temp pt}
%
% 
%    figcrop{<filename,w/o extension>} treats the first two labels as marking
%    the upper left and lower right corners of the figure. This is for
%    positioning purposes only. The figure may extend beyond the corners.
%    The corner markers are not printed.
%
%
\def\figcrop#1{\par%  #1=filename
\openin\labeLfile=\figdir#1.lbl                                              
\global\read\labeLfile to\firstline\message{#1}               
\global\read\labeLfile to\temp%read overall dimensions                                     
\readdim{\ybot}
\readdim{\xleft}%               read upper left point
\readdim{\ytop}
\global\read\labeLfile to\justx%ignore
\global\read\labeLfile to\justy%ignore
\global\read\labeLfile to\labeL%ignore
\readdim{\pswidth}%            read lower right point
\global\advance\pswidth by -\xleft
\readdim{\psheight}
\global\advance\ybot by -\psheight
\global\advance\psheight by -\ytop
\global\read\labeLfile to\justx%ignore
\global\read\labeLfile to\justy%ignore
\global\read\labeLfile to\labeL%ignore                                    
\vbox to\psheight{\vfill
%%%
%%% NOTE: next line may have to be changed for your DVIPS driver %%%
\ifnum\system=1\special{postscript grestore newpath gsave}\fi %textures
\ifnum\system=2\special{postscript grestore newpath gsave}\fi %msdos
\ifnum\system=3
  %%  \special{" grestore newpath gsave}
                                                 \fi         %%unix:dvips
\ifnum\system=4\special{" grestore newpath gsave}\fi         %%unix:dvips,scaled
\ifnum\system=1
\hbox to \pswidth{\kern-\xleft\special{postscriptfile \figdir#1.\suffix }\hfil}\fi
                                                              %textures
\ifnum\system=2
\hbox to \pswidth{\kern-\xleft\special{ps: plotfile \figdir#1.\suffix }\hfil}\fi
                                                              %mdos 
\ifnum\system=3
\hbox to \pswidth{\kern-\xleft\special{psfile=\figdir#1.\suffix }\hfil}\fi
                                                             %unix:dvips 
\ifnum\system=4
\hbox to \pswidth{\kern-\xleft\special{psfile=\figdir#1.\suffix hscale=120 vscale=120}\hfil}\fi
                                                             %unix:dvips,scaled
\ifnum\system=5
\hbox to \pswidth{\kern-\xleft\special{psfile="\figdir#1.\suffix"}\hfil}\fi %orphee
\ifnum\system=6
   \xdvikwid=\pswidth
   \xdvikht=\psheight
   {\global\divide\xdvikwid by \pspoints}
   {\global\divide\xdvikht by \pspoints}
   \rwi=\xdvikwid
    {\global\multiply\rwi by 10}
\hbox to \pswidth{\kern-\xleft\special{psfile=\figdir#1.\suffix\space  
                 llx=0\space lly=0\space
                 urx=\number\xdvikwid\space ury=\number\xdvikht\space 
                 rwi=\number\rwi}\hfil}\fi                   %xdvik
%%%
\vskip -\baselineskip
\vskip -\ybot 
\vskip-\psheight %                                     
\hbox to\pswidth  {\hss}%                                            
\parindent=0pt\offinterlineskip                                       
\vpos=0 pt%                                                              
\loop\readdim{\xcoord}                                 
\ifdim \xcoord < -999pt \doitfalse\else\doittrue\fi                        
\ifdoit \advance \xcoord by -\xleft
\readdim{\ycoord}
\advance \ycoord by -\ytop                              
\global\read\labeLfile to\justx                                       
\global\read\labeLfile to\justy                                       
\global\read\labeLfile to\labeL
\global\setbox\labox=\hbox{\labeL\hskip-0.3em}%    
\advance\vpos by-\ycoord                                              
\vskip-\vpos \vpos=\ycoord                                         
\hbox to\pswidth{\hskip\xcoord %                                 
\hbox to 0pt{\ifnum\justx>0\hss\fi%                                   
\vbox to0pt{%                                                         
\ifnum\justy<2\vss\fi%                                                
\copy\labox\kern0pt%  
\ifnum\justy>0\vss\fi}%                                               
\ifnum\justx<2\hss\fi}%                                               
\hss}%                                                                
\repeat%                                                              
\advance\vpos by-\psheight%                                           
\vskip-\vpos %                                                     
}\closein\labeLfile}
%
%
%     \figplace{<filename w/o extension>}{<hor shift>}{<vert shift>}
%     moves to the right by <hor shift> and down by <vert shift>
%     and then applies \figcrop
% 
\def\figplace#1#2#3{
\openin\labeLfile=\figdir#1.lbl
\ifeof \labeLfile
       \immediate\write16{***Can't find \figdir#1.lbl; Skipping it.***}
\else  \closein\labeLfile
       \null\hskip#2\raise #3 \hbox{\figcrop{#1}}
\fi
}
%
%
%     \figput{<filename w/o extension>}
%     
%     just applies \figcrop
% 
\def\figput#1{
\openin\labeLfile=\figdir#1.lbl
\ifeof \labeLfile
       \immediate\write16{***Can't find \figdir#1.lbl; Skipping it.***}
\else  \closein\labeLfile
       \hbox{\figcrop{#1}}
\fi
}

%
%
%     \rfig{text}{<filename w/o extension>}{caption}
%    
% Macro for text on left side of page and figure on right side.
% Figure is captioned below.
%
\newdimen\textsize
\long\def\rfig#1#2#3{\par\smallskip
\openin\labeLfile=\figdir#2.lbl
\ifeof\labeLfile
      \immediate\write16{***Can't find \figdir#2.lbl; Skipping it.***}
\else  
      \global\read\labeLfile to\headerinfo
      \readdim{\pswidth}
      \closein\labeLfile

      \textsize=\hsize
      \advance\textsize by-\pswidth  
      \advance\textsize by-1pc
      \line{\vtop{\hsize=\textsize{#1}\vfil}
      \hfill\vtop{\hsize=\pswidth \vskip0pt
      \figcrop{#2}\smallskip 
     \line{\hfil#3\hfil}\smallskip}}
\fi
}

