\def\ftmagnification{1200}
\def\spacingNumerator{1}
\def\spacingDenominator{1}
\input jmacros
%\input figMac
\def\figdir{fig/}
\def\showintremarks{n}

\titlea{The Check Column}

Ordinary arithmetic errors are a big problem when you do row operations
by hand. There is a technique called ``the check column'' (that is modeled
after the ``parity bit'' in computer hardware design) which provides a
very effective way to catch mechanical errors. Here is an example which
illustrates the technique



\example{}{
The augmented matrix for the system of equations
$$\deqalign{
  2&x_1&+&&x_2&+3x_3&=\,&&1\cr
  4&x_1&+&5&x_2&+7x_3&=\,&&7\cr
 2&x_1&-&5&x_2&+5x_3&=\,&-&7\cr
}$$
is
$$
\left[\left.\matrix{2&1&3\cr  4&5&7\cr  2&-5 &5\cr}
\right|\matrix{1\cr7\cr-7\cr}\right]
$$
To implement a ``check column'' you tack onto the right hand side of the 
augmented matrix an additional column. Each entry in this check column is 
the sum of all the entries in the row of the augmented matrix that is to 
the left of the check column entry. For example, the top entry in the check
column is $2+1+3+1=7$.
$$
\left[\left.\matrix{2&1&3\cr  4&5&7\cr  2&-5 &5\cr}
\right|\matrix{1\cr7\cr-7\cr}\right]\matrix{7\cr 23\cr -5}
$$
 To use the check column you just perform the same row
operations on the check column as you do on the augmented matrix. After
each row operation you check that each entry in the check column is
still the sum of all the entries in the corresponding row of the augmented
matrix.

We now want to eliminate the $x_1$'s from equations (2) and (3). That is,
we want to make the first entries in rows 2 and 3 of the augmented matrix
zero. We can achieve this by subtracting two times row (1) from row (2) and 
subtracting row (1) from row (3).
$$
\matrix{(1)\cr (2)-2(1)\cr (3)-(1)}
\left[\left.\matrix{2&1&3\cr  0&3&1\cr 0&-6 &2\cr}
\right|\matrix{1\cr 5\cr-8\cr}\right]\matrix{7\cr 9\cr -12}
$$
Observe that the check column entry $9$ is the sum $0+3+1+5$ 
of the entries in the second row of the augmented matrix. If this were
not the case, it would mean that we made a mechanical error. Similarly
the check column entry $-12$ is the sum $0-6+2-8$.

We have now succeeded in eliminating all of the $x_1$'s from equations
(2) and (3). For example, row 2 now stands for the equation
$$
3x_2+x_3=5
$$
We next use equation (2) to eliminate all $x_2$'s from equation
(3).
$$
\matrix{(1)\cr (2)\cr (3)+2(2)}
\left[\left.\matrix{2&1&3\cr  0&3&1\cr 0&0 &4\cr}
\right|\matrix{1\cr5\cr2\cr}\right]\matrix{7\cr 9\cr 6}
$$
We can now easily solve (3) for $x_3$, substitute the result back into (2) and
solve for $x_2$ and so on:
$$\meqalign{
(3)&\implies &&  4x_3& =2  && &\implies && x_3&=\half\cr
(2)&\implies &&  3x_2+\half&=5  && &\implies && x_2&=\sfrac{3}{2}\cr
(1)&\implies &&  2x_1+\sfrac{3}{2}+3\times\half&=1  && &\implies && x_1&=-1\cr
}$$
This last step is called ``backsolving''.

Note that there is an easy way to make sure that we have not made any mechanical
errors in deriving this solution --- just substitute the purported solution
$(-1,3/2,1/2)$ back into the original system:
$$\deqalign{
  2&(-1)&\,+\,&&\sfrac{3}{2}&+3\times\half&=\,&&1\cr
  4&(-1)&\,+&5\times &\sfrac{3}{2}&+7\times\half&=\,&&7\cr
 2&(-1)&\,-&5\times &\sfrac{3}{2}&+5\times\half&=\,&-&7\cr
}$$
and verify that each left hand side really is equal to its corresponding
right hand side.
}






















\end
